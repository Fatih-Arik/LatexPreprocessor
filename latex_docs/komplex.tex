\documentclass{article}
\usepackage{listings}

\define{NAME}{Test}
\define{NAME}
%\define{DEBUG}
\define{NENNER}{\frac(n(n+1), 42)}

\title{Mein Dokument}
\author{NAME}

\begin{document}
\maketitle

\section{Einleitung}
Hallo Welt! 
Mein Name ist \textbf{Max Mustermann} und ich nutze \textit{LaTeX}.
Und hier ist mehr Text

\section{Mathematik}
Inline-Mathe: $E = mc^2$  \\
Mathe: Hier ist ein Text über Mathe #math(\frac(NENNER, 2)) und hier geht es weiter.\\  

#math(\frac(1, \sqrt(2))) \\

#math(\pow(1, \sqrt(2))) \\

Hallo #math(\frac(1, 2)) Welt.  
#blockmath(\sqrt(\frac(4, \abs(3))))



\section(Include)
Hier ist ein bsp für \include{}


\section(Rekursiver Include)
Hier ist ein bsp für ein \include{} für eine Datei mit wiederum einem \include{}


\section{Quellcode}

#codeblock[c++]{
#define test 42
}


\section{ifdef}
\ifdef{DEBUG}
hier ist ein Bsp für Konditionale Makros!!!
\else
hier ist eine Abzweigung
\endif

\end{document}
